\documentclass[11pt, a4paper, DIV=12]{scrartcl}

% useful packages 
\usepackage{mathtools}
\usepackage{physics}
\usepackage{graphicx}					  
\graphicspath{{figs/}}



\usepackage{amssymb}
\usepackage{amsmath}
\usepackage{hyperref}
\usepackage[separate-uncertainty=true]{siunitx}
\usepackage{xcolor}
\usepackage{braket} % easy braket notation
\usepackage{enumitem}
\usepackage{booktabs}
\usepackage{here}
\usepackage{cprotect}

\usepackage[backend=biber, sorting=none]{biblatex}
\bibliography{refs.bib}

% \numberwithin{equation}{section}

\title{Simulation of 1-Dimensional Ising Model}
\date{\today}
\author{Harilal Bhattarai \& Marcel}
\begin{document}
	\maketitle
The Ising Model is Mathematical expression of the ferro-magnetism is statistical mechanics.\\

This system of spins is immersed in a heat bath of constant temperature T and in an external magnetic field h. Its dynamics are governed by the Hamiltonian,
\begin{equation}
H(s)= -J \sum_{<x,y>}s_{x}s_{y} - h \sum_{x}s_{x}
\end{equation}
$ <x,y> $ denotes the nearest-neighbor pair x and y on the chain, J is a real number and we assume periodic boundary conditions.
\end{document}
